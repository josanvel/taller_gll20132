% !TEX TS-program = pdflatex
% !TEX encoding = UTF-8 Unicode

% This is a simple template for a LaTeX document using the "article" class.
% See "book", "report", "letter" for other types of document.

\documentclass[11pt]{article} % use larger type; default would be 10pt

\usepackage[utf8]{inputenc} % set input encoding (not needed with XeLaTeX)

%%% Examples of Article customizations
% These packages are optional, depending whether you want the features they provide.
% See the LaTeX Companion or other references for full information.

%%% PAGE DIMENSIONS
\usepackage{geometry} % to change the page dimensions
\geometry{a4paper} % or letterpaper (US) or a5paper or....
% \geometry{margin=2in} % for example, change the margins to 2 inches all round
% \geometry{landscape} % set up the page for landscape
%   read geometry.pdf for detailed page layout information

\usepackage{graphicx} % support the \includegraphics command and options

% \usepackage[parfill]{parskip} % Activate to begin paragraphs with an empty line rather than an indent

%%% PACKAGES
\usepackage{booktabs} % for much better looking tables
\usepackage{array} % for better arrays (eg matrices) in maths
%\usepackage{paralist} % very flexible & customisable lists (eg. enumerate/itemize, etc.)
\usepackage{verbatim} % adds environment for commenting out blocks of text & for better verbatim
\usepackage{subfig} % make it possible to include more than one captioned figure/table in a single float
% These packages are all incorporated in the memoir class to one degree or another...

%%% HEADERS & FOOTERS
\usepackage{fancyhdr} % This should be set AFTER setting up the page geometry
\pagestyle{fancy} % options: empty , plain , fancy
\renewcommand{\headrulewidth}{0pt} % customise the layout...
\lhead{}\chead{}\rhead{}
\lfoot{}\cfoot{\thepage}\rfoot{}

%%% SECTION TITLE APPEARANCE
\usepackage{sectsty}
\allsectionsfont{\sffamily\mdseries\upshape} % (See the fntguide.pdf for font help)
% (This matches ConTeXt defaults)

%%% ToC (table of contents) APPEARANCE
\usepackage[nottoc,notlof,notlot]{tocbibind} % Put the bibliography in the ToC
\usepackage[titles,subfigure]{tocloft} % Alter the style of the Table of Contents
\renewcommand{\cftsecfont}{\rmfamily\mdseries\upshape}
\renewcommand{\cftsecpagefont}{\rmfamily\mdseries\upshape} % No bold!

%%% END Article customizations

\usepackage[spanish]{babel}
\usepackage{listings} 
%%% The "real" document content comes below...

\title{Investigación de Lenguajes - Pascal}
\author{Javier Tibau}
%\date{} % Activate to display a given date or no date (if empty),
         % otherwise the current date is printed 

\begin{document}
\maketitle
%\tableofcontents % No hace falta un TOC en un artículo corto

\section{Introducción}
asdf

\section{Características}
\section{Historia}
\begin{figure}[htbp]
	\begin{center}
		\includegraphics[width=.60\textwidth]{./imagenes/python.png}
		\caption{Logo}
		\label{Logo}
	\end{center}
\end{figure}

Python fue lanzado por primera vez en 1991, desarrolado por Guido van Rossum, un programador de origen holandés que desarrolló este lenguaje de programación a finales de los años 80 para el Centro para las Matemáticas y la Informática de los Países Bajos que buscaba un lenguaje de programación para ser utilizado bajo el sistema operativo Amoeba de Andrew S. Tanenbaum que fuese capaz de sustituir al lenguaje ABC.
\\\\
Python es un proyecto de codigo abierto, administrado por la Python Software Foundation.
\\\\
Python es un lenguaje de programación de alto nivel que fue diseñado con una sintaxis muy limpia que permitiese obtener códigos que fuesen fáciles de leer, es multiplataforma y soporta orientación a objetos, programación imperativa e, incluso, programación funcional.
\\\\
Se puede utilizar para muchos tipos de desarollo de software. El proposito del diseño del lenguaje Pyhton hace hincapie en  la productividad del programador y legibilidad del codigo.
\\\\
Su nombre fue inspirado el la seria THE MONTY PYTHON de la BBC de Londres.
\\\\
Hoy en dia, Python es mantenido por un numeroso grupo de voluntarios en todo el mundo.


\begin{figure}[htbp]
	\begin{center}
		\includegraphics[width=.50\textwidth]{./imagenes/python1.png}
		\caption{python}
		\label{python}
	\end{center}
\end{figure}

\section{Tutorial de Instalación}
\section{Hola Mundo y otros Programas Introductorios}

\lstset{language=Pascal}          % Set your language (you can change the language for each code-block optionally)

\begin{lstlisting}[frame=single]  % Start your code-block
for i:=maxint to 0 do
begin
{ do nothing }
end;
Write('Case insensitive ');
Write('Pascal keywords.');
\end{lstlisting}



\end{document}
